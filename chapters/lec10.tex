
\noindent A Hilbert space is a vector space $\mathcal{H}$ with an inner product $\langle | \rangle$ which induces a topology such that $\mathcal{H}$ is complete with respect to $\langle | \rangle$. \\

\noindent How do we attach a Hilbert space to a conformal field theory? \\

\noindent In other words: ``What are the states?'', ``What do the states correspond to?'', ``What are the observables and what do they correspond to?'', ``How do symmetries act on the Hilbert space (presumably unitarily)?''. \\

\noindent We can use Wightman functions to build $\mathcal{H}$ for ``good'' quantum field theories. By ``good'', we mean that if we take data, namely $n$-point correlation functions, from a QFT and use them to build a (linear) vector space with an inner product, then we would eventually find out that we actually have a Hilbert space. \\

\noindent The \textit{Wightman functions} are defined by

\begin{equation}
w_n \equiv \bra{0} \hat{\phi}_1 (x_1) \dots \hat{\phi} (x_n) \ket{0}.
\end{equation}

\noindent And define the ket vector by fields operating on the vacuum state

\begin{equation}
\ket{x_1 \dots x_n} \equiv \hat{\phi} (x_1) \dots \hat{\phi} (x_n) \ket{0}
\end{equation}

\noindent Such that the inner product, assuming the fields are self-adjoint distribution-valued objects, is

\begin{equation}
\braket{x_1 \dots x_n | y_1 \dots y_m } = \bra{0} \hat{\phi} (x_1) \dots \hat{\phi} (x_n)  \hat{\phi} (y_1) \dots \hat{\phi} (y_m) \ket{0}.
\end{equation}

\noindent So, we actually use the Wightman functions $w_n (x_1, \dots, x_n ; y_1, \dots, y_m)$ as ``smeared out'' versions of the operator products $\hat{\phi} (x_1) \dots \hat{\phi} (x_n)  \hat{\phi} (y_1) \dots \hat{\phi} (y_m)$, and define $f$ and $g$ as \textit{smearing functions}, with the inner product

\begin{equation}
\braket{ f_1 \dots f_n | g_1 \dots g_m } = \int dx_1 \dots dx_n dy_1 \dots dy_m \,\, \bar{f} (x_1) \dots \bar{f} (x_n) g(y_1) \dots g(y_m) w(x_1, \dots, x_n ; y_1, \dots, y_m).
\end{equation}

\noindent This is roughly what is happening in today's argument. \\

\noindent So, we first denote the vacuum state of a conformal field theory by $\ket{0}$, and then associate higher states with operators. \\

\noindent \textbf{Digression: Quantum Information Theory} \\

\noindent Suppose we have the $d$-dimensional complex numbers as a Hilbert space $\mathcal{H} = \mathbb{C}^d$ and $d \times d$ matrices as operators on the space $A \in \mathcal{B} (\mathcal{H})$. Then we can associate a quantum state in $\mathcal{H} \otimes \mathcal{H}$ to $A$ via the maximally entangled state

\begin{equation}
\ket{\Phi^+} = \sum_{j=1}^d \frac{1}{\sqrt{d}} \ket{jj} \text{ and } \ket{A} \equiv (A \otimes \mathbb{I}) \ket{\Phi^+}.
\end{equation}

\noindent \textbf{End Digression.} \\

\noindent Let $\hat{A} (z, \bar{z})$ be a field associated to the ``in'' state

\begin{align}
\ket{A_{in}} &\equiv \lim_{\sigma^0 \rightarrow -\infty} \hat{A} (\sigma^0, \sigma^1) \ket{0} \\
&= \lim_{\sigma^0 \rightarrow -\infty} e^{\sigma^0 \hat{H}} \hat{A} (0, \sigma^1) \ket{0} \\
&= \lim_{z, \bar{z} \rightarrow 0} \hat{A} (z, \bar{z}) \ket{0}.
\end{align}

\noindent Since we compactified space (to the Riemann sphere), all points at infinity are the same point, and a single state coming from $-\infty$ can be associated to the operator $\hat{A}$. \\

\noindent Now identify an ``out'' state, as dual to the ``in'' state, as a bra vector

\begin{equation}
\bra{A_{out}} \equiv \lim_{w,\bar{w} \rightarrow 0} \bra{0} \hat{\tilde{A}} (w,\bar{w})
\end{equation}

\noindent Where $w=\frac{1}{z}$ is a conformal transformation of the complex coordinates ($z=0 \rightarrow w=\infty$), and the tilde operator is the conformally transformed operator with some multiplicative factors $\hat{\tilde{A}} (w,\bar{w}) = \dots \hat{A} (z, \bar{z})$. \\

\noindent For a primary field $\hat{A} (z, \bar{z}) = \hat{\phi} (z, \bar{z})$, recall that we have the transformation law

\begin{equation}
f: \,\,\,\, \hat{\phi} (w, \bar{w}) \rightarrow \hat{\phi} (f(w), \bar{f} (\bar{w})) (\partial_w f(w))^h (\partial_{\bar{w}} \bar{f} (\bar{w}))^{\bar{h}}.
\end{equation}

\noindent So, the ``out'' state for a primary field looks like

\begin{align}
\bra{\phi_{out}} &= \lim_{a,\bar{z} \rightarrow 0} \bra{0} \hat{\phi} \left( \frac{1}{z}, \frac{1}{\bar{z}} \right) \frac{1}{z^{2h}} \frac{1}{\bar{z}^{2\bar{h}}} \\
&\equiv \lim_{a,\bar{z} \rightarrow 0} \bra{0} \hat{\phi} (\bar{z}, z)^\dagger \\
&= \left( \lim_{a,\bar{z} \rightarrow 0} \hat{\phi} (\bar{z}, z) \ket{0} \right)^\dagger \\
\bra{\phi}_{out} &= (\ket{\phi_{in}})^\dagger
\end{align}

\noindent Where we defined the adjoint in the second line with the factors, and define the dual states. \\

\noindent Using these definitions, we get the inner product

\begin{align}
\braket{\phi_{out} | \phi_{in}} &= \lim_{z,\bar{z}, w,\bar{w} \rightarrow 0} \bra{0} \hat{\phi} (z, \bar{z})^\dagger \hat{\phi} (w, \bar{w}) \ket{0} \\
&= \lim_{z,\bar{z}, w,\bar{w} \rightarrow 0}  \bar{z}^{-2h} z^{-2\bar{h}} \bra{0} \hat{\phi} \left( \frac{1}{\bar{z}}, \frac{1}{z} \right)  \hat{\phi} (w, \bar{w}) \ket{0} \\
&= \lim_{\xi, \bar{\xi} \rightarrow 0} \bar{\xi}^{2h} \xi^{2\bar{h}} \bra{0} \hat{\phi} (\bar{\xi}, \xi) \hat{\phi} (0,0) \ket{0}
\end{align}

\noindent Where we have sent $w, \bar{w} \rightarrow 0$ and set $\xi = \frac{1}{z}$ and $\bar{\xi} = \frac{1}{\bar{z}}$. This is because we have a formula for this correlation function, and it is independent of spacetime location $\xi$. \\

\noindent Morally, we can think of the vacuum state $\ket{0}$ as being like an entangled state, a bipartite state of an operator and its adjoint $\ket{\Phi^+}_{A\bar{A}}$, between the degrees of freedom at $-\infty$ and $\infty$. Then the ``in'' and ``out'' states can be thought of as

\begin{align}
\ket{A_{in}} &\rightarrow \hat{A} \otimes \mathbb{I} \ket{\Phi^+} \\
\ket{A_{out}} &\rightarrow \mathbb{I} \otimes \hat{\bar{A}} \ket{\Phi^+} .
\end{align}

\subsection*{Hilbert Space of CFT}

\noindent Now, let's build out the states to occupy the Hilbert space of our conformal field theory. \\

\noindent We have worked out the operators that correspond to global conformal transformations and form a $SL(2,\mathbb{C})$ subalgebra : $\{\hat{L}_{-1}, \hat{L}_0, \hat{L}_1 \}$. Note that the stress-energy, or energy-momentum, tensor generates spacetime deformations and these three operators are the global conformal transformations in infinitesimal form. \\

\noindent First, we must ensure that the vacuum state is invariant under $SL(2,\mathbb{C})$ by demanding that $\hat{T} (z) \ket{0}$ and $\hat{\bar{T}} (\bar{z}) \ket{0}$ are regular as $z,\bar{z} \rightarrow 0$. This requires that

\begin{align}
\hat{T} (z) \ket{0} = \sum_{n \in \mathbb{Z}} \,\, \hat{L}_n z^{-n-2} \ket{0} &\implies \hat{L}_n \ket{0} = 0, \,\, \forall \,\, n \ge -1 \\
&\,\,\,\,\, \text{and } \hat{\bar{L}}_n \ket{0} = 0, \,\, \forall \,\, n \ge -1.
\end{align}

\noindent Otherwise, the stress-energy tensor acting on the vacuum blows up as $z \rightarrow 0$. So, we have found that our subalgebra annihilates the vacuum. In other words, the vacuum state is invariant under global conformation transformations, the Virasoro generators, such that

\begin{equation}
\hat{L}_{-1} \ket{0} = \hat{L}_0 \ket{0} = \hat{L}_1 \ket{0} = 0.
\end{equation}

\noindent Now we wil use the OPE to show that the primary fields, our ``in'' states, are eigenstates of the Hamiltonian $\hat{H} = \hat{L}_0 + \hat{\bar{L}}_0$. Consider the commutator

\begin{align}
[ \hat{L}_n, \hat{\phi} (w, \bar{w}) ] &= \frac{1}{2 \pi i} \oint_W dz \,\, z^{n+1} \hat{T} (z) \hat{\phi} (w, \bar{w}) \\
&= \frac{1}{2 \pi i} \oint_W dz \,\, z^{n+1} \left( \frac{h}{(z-w)^2} \hat{\phi} (w, \bar{w}) + \frac{1}{z-w} \partial_w \hat{\phi} (w, \bar{w}) + \text{regular} \right) \\
[ \hat{L}_n, \hat{\phi} (w, \bar{w}) ] &= h (n+1) w^n \hat{\phi} (w, \bar{w}) + w^{n+1} \partial_w \hat{\phi} (w, \bar{w}).
\end{align}

\noindent Where the contour $W$ is a tight circle around the point $w$, and, since $z \rightarrow w$, we need only to analyze the singular behaviour of this expression, and the regular part is tossed out. \\

\noindent Define the asymptotic ``in'' state by the conformal weights $h$ and $\bar{h}$

\begin{equation}
\ket{h,\bar{h}} \equiv \hat{\phi} (0,\bar{0}) \ket{0}.
\end{equation}

\noindent Then we can work out the eigenvalues of the Virasoro generator by applying the above commutation relation to the defined ``in'' state

\begin{equation}
\hat{L}_0 \ket{h,\bar{h}} = \hat{L}_0 \hat{\phi} (0,\bar{0}) \ket{0} = [\hat{L}_0, \hat{\phi} (0,\bar{0}) ] \ket{0} = h \ket{h, \bar{h}}.
\end{equation}

\noindent Similarly for the dual Virasoro generator

\begin{equation}
\hat{\bar{L}}_0 \ket{h, \bar{h}} = \bar{h} \ket{h, \bar{h}} .
\end{equation}

\noindent And

\begin{equation}
\bar{L}_n \ket{h, \bar{h}} = \hat{\bar{L}}_n \ket{h, \bar{h}} = 0, \forall \,\, n>0.
\end{equation}

\noindent Excited states are created by expanding in Fourier modes of the primary field

\begin{equation}
\hat{\phi} (z, \bar{z}) = \sum_{m,n \in \mathbb{Z}} \,\, z^{-m+h}  \bar{z}^{-n+\bar{h}} \hat{\phi}_{m,n}.
\end{equation}

\noindent Inverting this to get

\begin{equation}
\hat{\phi}_{m,n} = \left(\frac{1}{2\pi i} \oint dz \, z^{m+h+1} \right) \left( \frac{1}{2\pi i} \oint d\bar{z} \, \bar{z}^{n+\bar{h}+1} \right) \hat{\phi} (z, \bar{z}).
\end{equation}

\noindent Note that on a real surface, $z=\bar{z}$, and $\hat{\phi}^\dagger_{m,n} = \hat{\phi}_{-m,-n}$. \\

\noindent For the ``in'' and ``out'' states to be well-defined, it is required that the modes annihilate the ground state for

\begin{equation}
\hat{\phi}_{m,n} = 0, \forall \,\, m > -h \text{ and } n >-\bar{h}.
\end{equation}

\noindent Note that we can drop the dependence on $\bar{z}$, since everything separates and we can reconstruct the $\bar{z}$-dependence from the $z$-dependence, such that

\begin{equation}
\hat{\phi} (z,\bar{z}) \rightarrow \hat{\phi} (z) = \sum_{n \in \mathbb{Z}} \, z^{-n-h} \hat{\phi}_n.
\end{equation}

\noindent And define the holomorphic modes by

\begin{equation}
\hat{\phi}_n \equiv \frac{1}{2\pi i} \oint dz \, z^{n+h-1} \hat{\phi} (z).
\end{equation}

\noindent Putting this together with the commutator above (of $\hat{L}_n$ and $\hat{\phi}(w,\bar{w})$) we get the commutation relation

\begin{equation}
[ \hat{L}_n, \hat{\phi}_m ] = (n(h-1)-m) \hat{\phi}_{m+n}.
\end{equation}

\noindent And we see that these operators act like ladder (raising/lowering) operators. For example, $\hat{\phi}_{-m}$ increases the conformal dimension by $m$. It is worth noting that $\hat{L}_{-m}$ also raises the conformal dimension

\begin{equation}
[ \hat{L}_0, \hat{L}_{-m} ] = m \hat{L}_{-m}.
\end{equation}

\noindent Now, we can apply these modes to the vacuum and build a big, intricate Hilbert space with vectors for each mode and inner products, but we'll just focus on the states constructed by applying $\hat{L}_n$ to the primary fields (``in'' state), and build a subspace that is invariant under global conformal transformations. \\

\noindent Let $\ket{h, \bar{h}} = \hat{\phi} (0, \bar{0}) \ket{0}$ be an asymptotic ``in'' state, and define for negative indices the \textit{descendant fields}

\begin{equation}
\hat{L}_{-k_1} \dots \hat{L}_{-k_n} \ket{h}, \text{where } 1 \le k_1 \le \dots \le k_n.
\end{equation}

\noindent Enacting $\hat{L}_0$ on the descendant field yields the eigenvalue $h' = h + k_1 + \dots + k_n \equiv h + N$, where we call $N = \sum_{j=1}^n k_j$ the \textit{level}. \\

\noindent The subspace generated by $\ket{h}$ and the descendant fields forms an irreducible representation, a module, of the Virasoro algebra called a \textit{Verma module}, or \textit{conformal family}, of the primary field. The state $\ket{h}$ behaves as the highest state vector of the Virasoro algebra. \\

\noindent We can work out some of the relationship between the central charge $c$ and the conformal weight $h$ by using a few facts about the Virasoro algebra

\begin{align}
&\hat{L}_n^\dagger = \hat{L}_{-n}, \\
&\hat{L}_0 \ket{h} = h \ket{h}, \\
&\hat{L}_n \ket{h} = 0, \forall \,\, n>0, \\
&[ \hat{L}_n , \hat{L}_m ] = (n-m) \hat{L}_{m+n} + \frac{c}{12} (n^3 - n) \delta_{m+n,0},
\end{align}

\noindent And considering the commutator

\begin{align}
\bra{h} [ \hat{L}_{-n}^\dagger, \hat{L}_{-n} ] \ket{h} &= \bra{h} [ \hat{L}_n, \hat{L}_{-n} ] \ket{h} \\
&= 2n \bra{h} \hat{L}_0 \ket{h} + \frac{c}{12} (n^3 - n) \braket{h|h} \\
&= \left( 2nh + \frac{c}{12} (n^3 - n) \right) \braket{h|h}.
\end{align}

\noindent From this, we observe that as $n \rightarrow \infty$ the only nontrivial representations of the Virasoro algebra are given by $c>0$. \\

\noindent For $n=1$, the only nontrivial representations are given by $h \ge 0$. \\

\noindent If $c=0$ and $h=0$, we get a trivial representation. \\

\noindent For arbitrary $h$, we look at the Gram matrix of $\hat{L}_{-2n} \ket{h}$ and $\hat{L}_{-n} \ket{h}$ and take the determinant

\begin{equation}
\text{det}(\text{Gram}) = 4n^3 h^2 (4h - 5n).
\end{equation}

\noindent As $n \rightarrow \infty$, the determinant becomes negative, and yields a trivial representation.
